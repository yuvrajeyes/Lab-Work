\documentclass[11pt]{article}
\usepackage{latexsym}
\usepackage{amsmath}
\usepackage{amssymb}
\usepackage{amsthm}
\usepackage{epsfig}
\usepackage{graphicx}
\usepackage{algorithm}
\usepackage{algpseudocode}
\usepackage[ruled,vlined]{algorithm2e}

\newcommand{\handout}[5]{
  \noindent
  \begin{center}
  \framebox{
    \vbox{
      \hbox to 5.78in { {\bf CSE-504: Mathematics For Computer Science } \hfill #2 }
      \vspace{4mm}
      \hbox to 5.78in { {\Large \hfill #5  \hfill} }
      \vspace{2mm}
      \hbox to 5.78in { {\em #3 \hfill #4} }
    }
  }
  \end{center}
  \vspace*{4mm}
}

\newcommand{\assignment}[4]{\handout{#1}{#2}{#3}{Submitted By: #4}{Assignment #1}}

\newtheorem{theorem}{Theorem}
\newtheorem{corollary}[theorem]{Corollary}
\newtheorem{lemma}[theorem]{Lemma}
\newtheorem{observation}[theorem]{Observation}
\newtheorem{proposition}[theorem]{Proposition}
\newtheorem{definition}[theorem]{Definition}
\newtheorem{claim}[theorem]{Claim}
\newtheorem{fact}[theorem]{Fact}
\newtheorem{assumption}[theorem]{Assumption}
\newtheorem{case}[theorem]{Case}
\newtheorem{problem}[theorem]{Problem}

% 1-inch margins, from fullpage.sty by H.Partl, Version 2, Dec. 15, 1988.
\topmargin 0pt
\advance \topmargin by -\headheight
\advance \topmargin by -\headsep
\textheight 8.9in
\oddsidemargin 0pt
\evensidemargin \oddsidemargin
\marginparwidth 0.5in
\textwidth 6.5in

\parindent 0in
\parskip 1.5ex
%\renewcommand{\baselinestretch}{1.25}

\begin{document}

\assignment{1 --- 12/02/2024}{2023-2024}{Prof.\ Dr.\ Obbattu Sai Lakshmi Bhavana}{Yuvraj Singh (23072021)}



\framebox[\linewidth]{
    \begin{minipage}{0.98\linewidth}
        \begin{problem}
            Consider a tank of capacity $500$ liters, initally empty, equipped with an input pipe and an outlet pipe, each controlled by respective buttons. Upon pressing the outlet button, $91$ liters of water is deducted from the tank, while
            pressing the inlet button adds $143$ liters of water. Determine the number of times the inlet button (i) and the outlet button (o) should be pressed to ensure the tank contains precisely $13$ liters of water at the end. Outlet button pressing when tank is empty will not be considered.
        \end{problem}
        \textbf{Solution:} The Extended Euclidean Algorithm can be used to find the number of inlet buttons (i) and the number of outlet buttons(o) such that $a.i + b.o = gcd(a, b)$. \\ Here, \vspace{-0.5em}
        \begin{equation}
            143i_0 - 91o_0 = gcd(143, 91) \vspace{-0.5em}
        \end{equation}
        Since capacity of the tank is limited to $500$ liters. \vspace{-1em}
        \begin{equation}
            143i - 91o \leq 500 \vspace{-1em}
        \end{equation}
        \vspace{-1em}
        Steps of the extended Euclidean algorithm:
        \begin{align*}
            \text{Step 1:} & \quad 143 = (-1) \times -91 + 52 \\
            \text{Step 2:} & \quad -91 = (-1) \times 52 - 39 \\
            \text{Step 3:} & \quad 52 = (-1) \times -39 + 13 \\
            \text{Step 4:} & \quad -39 = (-3) \times \underbrace{13}_{\text{gcd}} + 0 
        \end{align*}

        \vspace{-1.25em}
        Hence, equation (1) becomes \vspace{-0.5em}
        \begin{equation}
            143i - 91o = 13 \vspace{-0.5em}
        \end{equation} \vspace{-1em}
        Expressing the remainders in terms of the previous remainders:
        \begin{align*}
            \text{Step 3:} & \quad 13 = 52 - 1 \times 39 \\
            \text{Step 2:} & \quad 39 = 91 - 1 \times 52 \\
            \text{Step 1:} & \quad 52 = 143 - 1 \times 91 
        \end{align*}

        \vspace{-1.25em}
        Combining these: \vspace{-1.15em}
        $$ 39 = 1 \times 91 - 1 \times (143 - 1 \times 91) \implies 39 = 2 \times 91 - 1 \times 143 $$ \vspace{-2em}
        $$ 13 = 1 \times (143 - 1 \times 91) - 1 \times (2 \times 91 - 1 \times 143)$$ \vspace{-1.75em}
        $$ 13 = 2 \times 143 - 3 \times 91 $$ 
        
        \vspace{-0.5em}
        Therefore comparing these with equation (3), we get $i_0 = 2$ and $o_0 = 3$ \vspace{-1em}
        
        $$ i = i_0 + k \times 143 \equiv 2 + k \times 91$$ \vspace{-1.75em}
        $$ o = o_0 + k \times 91 \equiv -3 + k \times + 143$$ 
        
        \vspace{-0.75em}
        where k is an positive integer. 
        
        From equation (2), \vspace{-0.5em}
        $$ 143 \times (2 + k \times 91) - 91 \times (-3 + k \times 143) \leq 500 $$ 
        \vspace{-1.75em}
        $$ k \leq 130 / 13013 $$

        Since k is an integer therefore k = 0.

        Therefore the possible value of i = 2 and o = 3. \\

        Thus the number of times the inlet button (i) should be pressed is 2 and the outlet button (o) should be pressed is 3 to ensure the tank contains precisely $13$ liters of water at the end.
    \end{minipage}
}


\framebox[\linewidth]{
    \begin{minipage}{0.98\linewidth}
        \begin{problem}
            Prove that If $G$ and $H$ are groups then $G \times H$ is a group.
        \end{problem}
        \textbf{Solution:} Let \( G \) and \( H \) be groups with respective identity elements \( e_G \) and \( e_H \). \( G \times H \) is defined as the set of ordered pairs \( (g, h) \) where \( g \in G \) and \( h \in H \), equipped with the binary operation \( (g_1, h_1) \cdot (g_2, h_2) = (g_1 \cdot g_2, h_1 \cdot h_2) \). \\ 
        
        Now, let's prove the group properties:\\
        
        1. \textbf{Closure:} For any $(g_1, h_1), (g_2, h_2) \in G \times H$, the product $(g_1, h_1) \cdot (g_2, h_2) = (g_1 \cdot g_2, h_1 \cdot h_2)$ is an element of $G \times H$, so closure is satisfied.\\
        
        2. \textbf{Associativity:} The binary operation in set or group is defined as coordinate-wise multiplication in $G \times H$, and since multiplication in $G$ and $H$ is associative, the operation in $G \times H$ is also associative.\\
        Let for any $(g_1, \ h_1), (g_2, \ h_2), (g_3, \ h_3) \in G \times H$, the product 
        \begin{align*}
            ((g_1, h_1) \cdot (g_2, h_2)) \cdot (g_3, h_3) &= ((g_1 \cdot g_2, h_1 \cdot h_2)) \cdot (g_3, h_3) \\
            &= (g_1 \cdot g_2 \cdot g_3, h_1 \cdot h_2 \cdot h_3) \\
            &= (g_1 \cdot (g_2 \cdot g_3), h_1 \cdot (h_2 \cdot h_3))  \quad \ \text{(associativity in \(G\) and \(H\))} \\
            &= (g_1, h_1) \cdot (g_2 \cdot g_3, h_2 \cdot h_3)  \quad \quad \ \text{(closure in \(G\) and \(H\))} \\
            &= (g_1, h_1) \cdot ((g_2, h_3) \cdot (g_3, h_3))  \quad \text{(closure in \(G\) and \(H\))}
        \end{align*}
        
        Hence $G \times H$ is associative.\\
        
        3. \textbf{Identity Element:} The identity element in $G \times H$ is $(e_G, e_H)$ because $(g, h) \cdot (e_G, e_H) = (g \cdot e_G, h \cdot e_H) = (g, h)$ and $(e_G, e_H) \cdot (g, h) = (e_G \cdot g, e_H \cdot h) = (g, h)$ for any $(g, h) \in G \times H$.\\
        
        4. \textbf{Inverses:} The inverse of $(g, h)$ in $G \times H$ is $(g^{-1}, h^{-1})$ because $(g, h) \cdot (g^{-1}, h^{-1}) = (g \cdot g^{-1}, h \cdot h^{-1}) = (e_G, e_H)$ and $(g^{-1}, h^{-1}) \cdot (g, h) = (g^{-1} \cdot g, h^{-1} \cdot h) = (e_G, e_H)$ for any $(g, h) \in G \times H$. \\
        Therefore, $G \times H$ is a group under the defined binary operation.\\ \\

        \(G \times H\) is abelian if and only if both \(G\) and \(H\) are abelian groups.

        To show this, let's consider \(G \times H\) with the binary operation defined as \((g_1, h_1) \cdot (g_2, h_2) = (g_1 \cdot g_2, h_1 \cdot h_2)\).
        
        Now, if \(G\) and \(H\) are abelian groups, then for any elements \(g_1, g_2 \in G\) and \(h_1, h_2 \in H\), we have:

        \vspace{-1em}
        \begin{align*}
        (g_1, h_1) \cdot (g_2, h_2) &= (g_1 \cdot g_2, h_1 \cdot h_2) \\
        &= (g_2 \cdot g_1, h_2 \cdot h_1) \quad \text{(Commutativity in \(G\) and \(H\))} \\
        &= (g_2, h_2) \cdot (g_1, h_1)
        \end{align*}
        
        Thus, the direct product \(G \times H\) is also abelian.
        
        In summary, \(G \times H\) is abelian if \(G\) and \(H\) are abelian groups.
    \end{minipage}
}

\framebox[\linewidth]{
    \begin{minipage}{0.98\linewidth}
        \begin{problem}
            What is the element in $\mathbb{Z}_{21}^*$ corresponding to (2, 3) in $\mathbb{Z}_{3}^* \times \mathbb{Z}_{7}^*$ under the given isomorphism?
        \end{problem}
        \textbf{Solution:}
        If we define an isomorphism from $\mathbb{Z}_3^* \times \mathbb{Z}_7^*$ to $\mathbb{Z}_{21}^*$, we need to find a suitable mapping. One common choice is to use the Chinese Remainder Theorem (CRT).
        
        Assuming the isomorphism $\phi: \mathbb{Z}_3^* \times \mathbb{Z}_7^* \rightarrow \mathbb{Z}_{21}^*$ defined by
        \[
        \phi(a, b) = \left[ a \cdot 7 \cdot 7^{-1} + b \cdot 3 \cdot 3^{-1} \right] \mod 21.
        \]

        Here, $7^{-1}$ is the modular inverse of $7$ modulo $3$, and $3^{-1}$ is the modular inverse of $3$ modulo $7$.

        $7^{-1} = 1$ and $3^{-1} = 5$ \quad (using extended Euclid's Algorithm)\\
        
        Now, let's find $\phi(2, 3)$:
        \begin{align*}
        \phi(2, 3) & = \left[ 2 \cdot 7 \cdot 7^{-1} + 3 \cdot 3 \cdot 3^{-1} \right] \mod 21 \\
        & = \left[ 2 \cdot 7 \cdot 1 + 3 \cdot 3 \cdot 5 \right] \mod 21 \\
        & = \left[ 14 + 45 \right] \mod 21 \\
        & = \left[ 59 \right] \mod 21 \\
        & = 17 \mod 21.
        \end{align*}
        
        So, under this isomorphism, the element in $\mathbb{Z}_{21}^*$ corresponding to $(2, 3)$ is $17$. 
    \end{minipage}
}

\framebox[\linewidth]{
    \begin{minipage}{0.98\linewidth}
        \begin{problem}
           Compute the final two (decimal) digits of $3^{1000}$.
        \end{problem}
        \textbf{Solution:} Since we have to compute the final two digit, we can use $3^{1000} \pmod{100}$. \\
        $\because$ $gcd(3, 100) = 1$, we can use Euler Theorem.\\
        
        By Euler's Theorem, if $a$ and $n$ are coprime positive integers, then $a^{\phi(n)} \equiv 1 \pmod{n}$, where $\phi(n)$ is Euler's totient function.
        \\
        
        In this case, let $n = 100$. The numbers coprime to 100 are those not divisible by 2 or 5. Therefore, $\phi(100) = 40$.
        
        So, according to Euler's Theorem:
        \[3^{\phi(100)} \equiv 1 \pmod{100}\]

        Now, we know that $a^{p} \pmod{n} \equiv a^{p\pmod{\phi(n)}} \pmod{n}$ \quad \text{(Consequence of Euler's Theorem)} 
        
        Therefore,  $3^{1000} \pmod{100}$, can be express as $3^{1000\pmod{40}} \pmod{100}$:
        \[3^{1000} \equiv 3^{1000\pmod{40}} \equiv 3^{0} \equiv 1 \pmod{100}\]
        
        Therefore, the final two decimal digits of $3^{1000}$ are 01.

    \end{minipage}
}

\framebox[\linewidth]{
    \begin{minipage}{0.98\linewidth}
        \begin{problem}
            Write the algorithm for efficient group exponentiation using the \textbf{square-and-multiply} technique.
        \end{problem}
        \textbf{Solution:}
        \begin{algorithm}[H]
          \caption{Group Exponentiation using Square-and-Multiply}
          \KwIn{base element $g$, exponent $e = \sum_{i=0}^{t} e_i 2^i$ with $e_i \in \{0, 1\}$ and $e_t = 1$}
          \KwOut{$g^e$}
          \textbf{Initialization:} $result = 1$\; \\ 
          
          \For{$i = t-1$ DOWNTO $0$}{
            $result = result^2$\; \\
            \If{$e_i = 1$}{
              $result = result * g$\;
            }
          }
          \textbf{RETURN} $(result)$\;
        \end{algorithm}
        \textbf{Time Complexity: O(t)} or \textbf{O(loge)}
    \end{minipage}
}

\framebox[\linewidth]{
    \begin{minipage}{0.98\linewidth}
        \begin{problem}
            Analyze the complexity of \textbf{Euclid’s algorithm} in terms of its input size.
        \end{problem}
        \textbf{Solution:} 
        The Euclidean algorithm is used for finding the greatest common divisor (GCD) of two integers. Let's denote the input size by $n$, where $n$ is the number of bits required to represent the larger of the two integers.\\
        
        The complexity of Euclid's algorithm can be analyzed in terms of the number of iterations it takes to reach the GCD. The algorithm uses the division remainder in each step until the remainder becomes zero. Therefore, the number of iterations depends on the size of the numbers involved.\\
        
        Let $a$ and $b$ be the two input integers. \textbf{The worst-case scenario occurs when $a$ and $b$ are consecutive Fibonacci numbers.} In this case, the number of iterations $k$ is proportional to the smaller of $a$ and $b$ and can be approximated by: \vspace{-0.75em}
        
        \[ k \approx \log(\min(a, b)) \]

        Now, since $n$ is the number of bits in the larger of the two numbers, we can express $a$ and $b$ in terms of $n$: \vspace{-1.5em}
        
        \[ a, b \leq 2^n \]
        
        So, the complexity of Euclid's algorithm in terms of the input size $n$ can be expressed as: \vspace{-0.75em}
        
        \[ O(\log(2^n)) \]
        
        This simplifies to: \vspace{-0.75em}
        
        \[ O(n \cdot \log(2)) = O(n)\]

        As a result, the complexity of Euclid's algorithm is linear in the size of the input where input is the number of bits required to represent the larger of the two intergers.
    \end{minipage}
}

\framebox[\linewidth]{
    \begin{minipage}{0.98\linewidth}
        \begin{problem}
            Compute $101^{4,800,000,002} \pmod{35}$ .
        \end{problem}
        \textbf{Solution:} Since 101 and 35 are coprrime. Therefore to compute $101^{4,800,000,002} \pmod{35}$ we can use the  Euler's Theorem.
        $$
        101^{4,800,000,002} \pmod{35} \equiv (2 \times 35 + 31)^{4,800,000,002} \pmod{35} \equiv 31^{4,800,000,002} \pmod{35}
        $$
        \begin{align*}
            31^{4,800,000,002} \pmod{35} \equiv 31^{4,800,000,002 \ mod \ \phi(35)} \pmod{35}
        \end{align*}
        $$
        31^{4,800,000,002 \ mod \ 24} \pmod{35} \equiv 31^{2} \pmod{35} \equiv 961 \pmod{35}
        $$
        $$961 \pmod{35} = 16 \pmod{35}$$
            
        Therefore, $101^{4,800,000,002} \equiv 16 \pmod{35}$.
    \end{minipage}
}

\framebox[\linewidth]{
    \begin{minipage}{0.98\linewidth}
        \begin{problem}
            Compute $29^{100} \pmod{35}$ and $18^{25} \pmod{35}$ using \textbf{Chinese remainder theorem.}
        \end{problem}
        \textbf{Solution:}
        To compute $29^{100} \pmod{35}$ and $18^{25} \pmod{35}$ using the Chinese Remainder Theorem (CRT), we need to express the modular arithmetic in terms of the prime factorization of the modulus $35$, i.e., $35 = 5 \times 7$.

        \vspace{-1em}
        \begin{enumerate}
            \item \textbf{Express the numbers in terms of prime factorization:} \vspace{-0.5em}
            \begin{align*}
                29^{100} &\equiv x \pmod{35} \\
                18^{25} &\equiv y \pmod{35}
            \end{align*}

            \vspace{-1em}
            \item \textbf{Apply CRT:}
            The Chinese Remainder Theorem states that if $n = p \times q$, then for any $a$ and $b$, the solution to the system of congruences:
            \vspace{-1em}
            \begin{align*}
                x &\equiv a \pmod{p} \\
                x &\equiv b \pmod{q} \vspace{-1em}
            \end{align*} 
            \vspace{-2.5em}
            
            is given by:
            \vspace{-1em}
            \[
                x \equiv (a \times q \times q^{-1} \pmod{n}) + (b \times p \times p^{-1} \pmod{n}) \pmod{n}
                \vspace{-1em}
            \]
            
            where $q^{-1}$ and $p^{-1}$ are the modular inverses of $q$ and $p$ modulo $p$ and $q$ respectively.

            \item \textbf{Apply CRT for $29^{100} \pmod{35}$:}
            Prime factorization of $35 = 5 \times 7$. \vspace{-1em}
            \begin{align*}
                29^{100} &\equiv x \pmod{35} \implies
                \begin{cases}
                    x \equiv 29^{100} \pmod{5} \\
                    x \equiv 29^{100} \pmod{7}
                \end{cases}
            \end{align*}
            \vspace{-2.25em}
            
            Calculate each part separately:
            \begin{align*}
                29^{100} &\equiv 4^{100} \equiv 4^{20 \times 5} \equiv (4^5)^{20} \equiv (-1)^{20} \equiv 1 \pmod{5} \\
                29^{100} &\equiv 1^{100} \equiv 1 \pmod{7}
            \end{align*}
            \vspace{-2.5em}
            
            Apply CRT:
            \vspace{-1.25em}
            \begin{align*}
                x &\equiv (1 \times 7 \times 7^{-1} \pmod{35}) + (1 \times 5 \times 5^{-1} \pmod{35}) \pmod{35}
            \end{align*}

            \vspace{-1em}
            Calculate the modular inverses:

            \vspace{-2.25em}
            \begin{align*}
                7^{-1} &\equiv 3 \pmod{5}, \quad 5^{-1} \equiv 3 \pmod{7}
            \end{align*}

            \vspace{-1em}
            Substitute back into the CRT formula: \vspace{-0.5em}
            \begin{align*}
                x &\equiv (1 \times 7 \times 3) + (1 \times 5 \times 3) \pmod{35} \equiv 1 \pmod{35}
            \end{align*}

            \vspace{-1em}
            \item \textbf{Apply CRT for $18^{25} \pmod{35}$:}
            Prime factorization of $35 = 5 \times 7$. \vspace{-1em}
            \begin{align*}
                18^{25} &\equiv y \pmod{35} \implies
                \begin{cases}
                    y \equiv 18^{25} \pmod{5} \\
                    y \equiv 18^{25} \pmod{7}
                \end{cases}
            \end{align*}

            \vspace{-1em}
            Calculate each part separately: \vspace{-0.5em}
            \begin{align*}
                18^{25} &\equiv 3^{25} \equiv 3^{25 \ mod \ \phi(5)} \equiv 3^{25 \ mod \ 4} \equiv 3 \pmod{5} \\
                18^{25} &\equiv 4^{25} \equiv 4^{25 \ mod \ \phi(7)} \equiv 4^{25 \ mod \ 6} \equiv 4 \pmod{7} 
            \end{align*}

            \vspace{-1.5em}
            Apply CRT: \vspace{-1em}
            \begin{align*}
                y &\equiv (3 \times 7 \times 7^{-1} \pmod{35}) + (4 \times 5 \times 5^{-1} \pmod{35}) \pmod{35}
            \end{align*}

            \vspace{-1em}
            Calculate the modular inverses: \vspace{-0.75em}
            \begin{align*}
                5^{-1} &\equiv 3 \pmod{7}, \quad 7^{-1} \equiv 3 \pmod{5}
            \end{align*}

            \vspace{-1.2em}
            Substitute back into the CRT formula: \vspace{-0.75em}
            \begin{align*}
                y &\equiv (3 \times 7 \times 3) + (4 \times 5 \times 3) \pmod{35} \equiv 18 \pmod{35}
            \end{align*}
        \end{enumerate}

        \vspace{-1em}
        Therefore, $29^{100} \equiv 1 \pmod{35}$ and $18^{25} \equiv 18 \pmod{35}$.
        
    \end{minipage}
}

\framebox[\linewidth]{
    \begin{minipage}{0.98\linewidth}
        \begin{problem}
            Compute the inverse of $197$ in $\mathbb{Z}_{1997}^*$. (Use Extended Euclidean Algorithm)
        \end{problem}
        \textbf{Solution:} To find the inverse of $197$ in $\mathbb{Z}_{1997}^*$ using the Extended Euclidean Algorithm, we need to solve the linear Diophantine equation: \vspace{-1em}
        $$197x + 1997y = 1$$
        \vspace{-3em}
        \begin{align*}
            \text{Step 1:} & \quad 1997 = 10 \times 197 + 27 \\
            \text{Step 2:} & \quad 197 = 7 \times 27 + 8 \\
            \text{Step 3:} & \quad 27 = 3 \times 8 + 3 \\
            \text{Step 4:} & \quad 8 = 2 \times 3 + 2 \\
            \text{Step 5:} & \quad 3 = 1 \times 2 + 1
        \end{align*}
        
        \vspace{-0.5em}
        Now, we work backward to express each remainder in terms of the previous remainders:
        \vspace{-0.5em}
        \begin{align*}
            \text{Step 5:} & \quad 1 = 3 - 1 \times 2 \\
            \text{Step 4:} & \quad 1 = 3 - 1 \times (8 - 2 \times 3) \\
            & \quad \ = 3 \times 3 - 1 \times 8 \\
            \text{Step 3:} & \quad 1 = 3 \times (27 - 3 \times 8) - 1 \times 8 \\
            & \quad \ = 3 \times 27 - 10 \times 8 \\
            \text{Step 2:} & \quad 1 = 3 \times 27 - 10 \times (197 - 7 \times 27) \\
            & \quad = 73 \times 27 - 10 \times 197 \\
            \text{Step 1:} & \quad 1 = 73 \times (1997 - 10 \times 197) -  10 \times 197 \\
            & \quad = 73 \times 1997 - 740 \times 197
        \end{align*}

        Now, looking at the coefficient of $197$, we see that $(-740) \times 1997 \equiv 1 \pmod{197}$. Therefore, the inverse of $197$ in $\mathbb{Z}_{1997}^*$ is $-32 \equiv 1257 \pmod{1997}$.

        Therefore, $197^{-1} \equiv \textbf{1257} \pmod{1997}$.
    \end{minipage}
}

\framebox[\linewidth]{
    \begin{minipage}{0.98\linewidth}
        \begin{problem}
           Write algorithm for \textbf{Miller-Rabin primality test.}
        \end{problem}
        \textbf{Solution:}
        \begin{algorithm}[H]
            \caption{Miller-Rabin Primality Test}
            \KwIn{Integer $N > 2$ and parameter $t$}
            \KwOut{A decision as to whether $N$ is prime or composite} \\

            \text{} \\
            \If{$N$ is even or $N$ is a perfect power}{
              \Return "composite"\;
            }
            \ Compute $r \geq 1$ and $u$ odd such that $N - 1 = 2^ru$\;\\
            \For{$j = 1$ to $t$}{
               \ Choose $a \in \mathbb{Z}_N^*$\; \\
              \If{\(\text{gcd}(a, N) \neq 1\) or \((a^u \not\equiv \pm 1 \pmod{N})\) and \((a^{2^iu} \not\equiv -1 \pmod{N})\) for \(i \in \{1, \ldots, r - 1\}\)}{
                  \Return "composite"\;
              }
            }
            \Return "prime"\;
        \end{algorithm}
        \textbf{Time Complexity: O(t⋅(logN)^{3)}}
    \end{minipage}
}

\end{document}